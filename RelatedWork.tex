\section{Related Work}
A number of significant researches have already done in text classification in English and other language. Major works of text classification are Email classification, Research paper categorization, Detecting suspicious profiles etc.
But research on Bangla text classification is in preliminary stage still now. However some mentionable works have been done for Bangla Language Processing.
\vspace{0.2cm}

Hossain et al. describes Bengali document categorization based on word embedding and statistical learning approaches\cite{hossain2018automatic}. It categorizes document into nine predefined categories with mentionable accuracy. An Arabic text categorization system is developed using Naive Bayes in control environment dataset with good accuracy\cite{alsaleem2011automated}.
Krendzelak et al. describe a system which categorize text with machine learning and hierarchical structures by using tree based Naive Bayesian categorization process\cite{krendzelak2015text, chy2014bangla}. It performs with low accuracy due to training techniques and training feature extraction process.
\vspace{0.2cm}

S. Alami et al. describes about different techniques to detect suspicious profiles using text analysis within social media\cite{alami2015detecting}. A system for detecting suspicious email using enhanced feature selection is proposed but it has low accuracy because of not having enough dataset\cite{nizamani2013modeling}. Text Categorization of Turkish language using SVM is proposed which achieved better accuracy but due to large feature dimensions time complexity is large\cite{kaya2012sentiment}. Better result can be obtained by using clustering based approach \cite{ismail2014bangla, ahmad2016bengali} but a lot of problem exist with cluster-based solution. In our work, a system is developed and trained with different machine learning algorithms and overall accuracy of this algorithms is measured over our dataset.