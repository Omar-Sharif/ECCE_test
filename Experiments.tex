\section{Experiments}
For the successful implementation of any machine learning algorithm dataset is the key. It was a very challenging task for us to build a corpus which contains a large amount of suspicious and non suspicious text. We have collected non suspicious data from a pre-build corpus\cite{banglacorpus}, thanks to them. Suspicious data are collected from different online and offline resources.% Data stored in .txt format.
\par \vspace{0.3cm} 
We mark a text as suspicious if it has one of the following features,
\begin{itemize}
    \item Texts contain words which hurts our religious feelings.\vspace{0.2cm} 
    \item Texts which instigate people against government.\vspace{0.2cm} 
    \item Texts which instigate people against law enforcement agencies.\vspace{0.2cm} 
    \item Texts which motivate people in terrorist events.\vspace{0.2cm} 
    \item Texts which instigate a community without any reason.\vspace{0.2cm} 
    \item Texts which instigate our political parties. 
\end{itemize}
 \par \vspace{0.3cm}
 Most of our suspicious data about religion are collected from online blogs\cite{nastikya, dhormo, istishon}. Suspicious data about politics is collected from websites of different newspaper\cite{palo, kk, juga}. Data is also collected from different public pages of Facebook\cite{bash}. Table \ref{data} represents the statistics of data used for our model.
 
 \renewcommand{\arraystretch}{1.3}
\begin{table}[h!]
\begin{center}
\caption{Data Summary}
\begin{tabular}{|m{4.8cm} | m{3cm}|}
\hline
     Number of training documents & 1500 \\
\hline
     Number of testing documents & 500 \\
\hline
     Number of sentences & 8991\\
\hline 
     Number of words & 35964\\
\hline 
     Total unique words & 4295\\
\hline
\end{tabular}
\label{data}
\end{center}
\end{table}

In order to classify the texts, we have fed our collected documents to our classifier model. As dataset is collected manually it may have some inconsistency.