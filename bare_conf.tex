\documentclass[conference]{IEEEtran}
\hyphenation{op-tical net-works semi-conduc-tor}
\usepackage[bibstyle=ieee, citestyle=numeric-comp]{biblatex}
\usepackage{graphicx}
\usepackage{amsmath}
\usepackage{listings}
\usepackage{caption}
\usepackage{subcaption}
\usepackage{csquotes}
\usepackage{float}
\usepackage{color}
\usepackage{gensymb}
\usepackage{multirow}
\usepackage{array}
\usepackage{mathtools}
\usepackage[thinlines]{easytable}
\DeclarePairedDelimiter{\abs}{\lvert}{\rvert}
\usepackage[toc,page]{appendix}
\usepackage{setspace}
\usepackage{biblatex}
\addbibresource{ref.bib}
\captionsetup[figure]{labelfont={bf},labelformat={default},labelsep=period,name={Fig.}}
\addbibresource{ref.bib}
\renewcommand\IEEEkeywordsname{Keywords}

\begin{document}

\title{Suspicious Bangla Text Detection Using\\ Machine Learning}

\author{\IEEEauthorblockN{Omar Sharif}
\IEEEauthorblockA{Dept. of Computer Science \& Engineering\\
Chittagong University of Engineering and Technology\\
Chittagong-4349, Bangladesh\\
Email: omar.shaif1303@gmail.com}
\and
\IEEEauthorblockN{Mohammed Moshiul Hoque}
\IEEEauthorblockA{Dept. of Computer Science \& Engineering\\
Chittagong University of Engineering and Technology\\
Chittagong-4349, Bangladesh\\
Email: moshiulh@yahoo.com}}


\maketitle

%%%%%%% Abstract and Keyword section %%%%%%%
%%%%%%%%%%%%%%%%%%%%%%%%%%%%%%%%%%%%%%%%%%%%
\begin{abstract}
Suspicious Bangla text detection is a text classification problem of classifying Bangla text into suspicious and non suspicious category. In this paper we tried to find out the  performance of different statistical machine learning algorithm for detecting suspicious Bangla text. To the best of our knowledge, it is very first work on detecting suspicious Bangla text so we have to develop a corpus of suspicious Bangla text. This paper shows comparison of accuracy of different machine learning algorithm which will be helpful to others. The experimental result shows maximum accuracy of 92\% for Logistic Regression using 1500 training documents and 500 testing documents. The overall accuracy can be increased by increasing number of text documents in the dataset and considering semantic relation between words of a sentence.  
\end{abstract}
\vspace{0.3cm}
\begin{IEEEkeywords}
 Suspicious Bangla text, Text classification, Text classification methods, Bangla language processing, Machine learning. 
\end{IEEEkeywords}

\IEEEpeerreviewmaketitle

%%%%%%%%%% Introduction %%%%%%%%%%%%%%%%%%%%
%%%%%%%%%%%%%%%%%%%%%%%%%%%%%%%%%%%%%%%%%%%%
\section{Introduction}
Text classification is the task of assigning a text document into a set of predefined classes in an intelligent manner. Because of the rapid growth of online information, text classification has become more challenging and more important as well. Digitization has changed
the way we process and analyze information. There is an exponential increase in online availability of information. From web pages to emails, science journals, ebooks, learning content, news and social media are all full of textual data. Text classification performs an essential role in various applications that deals with organizing, classifying, searching and concisely representing a significant amount of information. Detecting suspicious text is typically a text classification problem where we have to classify a text as suspicious and not suspicious. Suspicious text detection is a kind of system where suspicious texts are identified by the keywords used in the text body. As most of our communications are text based, if we able to predict either a text is suspicious or not suspicious it will be very helpful for our law enforcement agencies to find the perpetrator and stop terrorist event. As far we know, no system is developed for detecting suspicious Bangla text. It is very important for the safety of Bangladeshi People to develop a system which can detect suspicious communications in Bangla. This motivated us to work in this area.

In this work, our original purpose is to develop a framework for detecting suspicious Bangla texts using supervised machine learning techniques. In this paper, we have applied Naïve Bayes Classifier\cite{yoo2015classification}, Support Vector Machine\cite{wei2012text, villmann2017can}, Logistic regression\cite{sharma2015active}, K-nearest neighbor algorithms (KNN)\cite{harisinghaney2014text}, Decision Trees\cite{chavan2014survey} to detect suspicious Bangla text and also find out the accuracy of this algorithms.

%%%%%%%%%% Related Work %%%%%%%%%%%%%%%%%%%%
%%%%%%%%%%%%%%%%%%%%%%%%%%%%%%%%%%%%%%%%%%%%
\section{\textbf{Related Work}}
A number of significant researches have already done in text classification in English and other language. Major works of text classification are Email classification, Research paper categorization, Detecting suspicious profiles etc.
But research on Bangla text classification is in preliminary stage still now. However some mentionable works have been done for Bangla Language Processing.
\vspace{0.2cm}

Hossain et al. describes Bengali document categorization based on word embedding and statistical learning approaches\cite{hossain2018automatic}. It categorizes document into nine predefined categories with mentionable accuracy. An Arabic text categorization system is developed using Naive Bayes in control environment dataset with good accuracy\cite{alsaleem2011automated}.
Krendzelak et al. describe a system which categorize text with machine learning and hierarchical structures by using tree based Naive Bayesian categorization process\cite{krendzelak2015text, chy2014bangla}. It performs with low accuracy due to training techniques and training feature extraction process.
\vspace{0.2cm}

S. Alami et al. describes about different techniques to detect suspicious profiles using text analysis within social media\cite{alami2015detecting}. A system for detecting suspicious email using enhanced feature selection is proposed but it has low accuracy because of not having enough dataset\cite{nizamani2013modeling}. Text Categorization of Turkish language using SVM is proposed which achieved better accuracy but due to large feature dimensions time complexity is large\cite{kaya2012sentiment}. Better result can be obtained by using clustering based approach \cite{ismail2014bangla, ahmad2016bengali} but a lot of problem exist with cluster-based solution. In our work, a system is developed and trained with different machine learning algorithms and overall accuracy of this algorithms is measured over our dataset.






































% An example of a floating figure using the graphicx package.
% Note that \label must occur AFTER (or within) \caption.
% For figures, \caption should occur after the \includegraphics.
% Note that IEEEtran v1.7 and later has special internal code that
% is designed to preserve the operation of \label within \caption
% even when the captionsoff option is in effect. However, because
% of issues like this, it may be the safest practice to put all your
% \label just after \caption rather than within \caption{}.
%
% Reminder: the "draftcls" or "draftclsnofoot", not "draft", class
% option should be used if it is desired that the figures are to be
% displayed while in draft mode.
%
%\begin{figure}[!t]
%\centering
%\includegraphics[width=2.5in]{myfigure}
% where an .eps filename suffix will be assumed under latex, 
% and a .pdf suffix will be assumed for pdflatex; or what has been declared
% via \DeclareGraphicsExtensions.
%\caption{Simulation results for the network.}
%\label{fig_sim}
%\end{figure}

% Note that the IEEE typically puts floats only at the top, even when this
% results in a large percentage of a column being occupied by floats.


% An example of a double column floating figure using two subfigures.
% (The subfig.sty package must be loaded for this to work.)
% The subfigure \label commands are set within each subfloat command,
% and the \label for the overall figure must come after \caption.
% \hfil is used as a separator to get equal spacing.
% Watch out that the combined width of all the subfigures on a 
% line do not exceed the text width or a line break will occur.
%
%\begin{figure*}[!t]
%\centering
%\subfloat[Case I]{\includegraphics[width=2.5in]{box}%
%\label{fig_first_case}}
%\hfil
%\subfloat[Case II]{\includegraphics[width=2.5in]{box}%
%\label{fig_second_case}}
%\caption{Simulation results for the network.}
%\label{fig_sim}
%\end{figure*}
%
% Note that often IEEE papers with subfigures do not employ subfigure
% captions (using the optional argument to \subfloat[]), but instead will
% reference/describe all of them (a), (b), etc., within the main caption.
% Be aware that for subfig.sty to generate the (a), (b), etc., subfigure
% labels, the optional argument to \subfloat must be present. If a
% subcaption is not desired, just leave its contents blank,
% e.g., \subfloat[].


% An example of a floating table. Note that, for IEEE style tables, the
% \caption command should come BEFORE the table and, given that table
% captions serve much like titles, are usually capitalized except for words
% such as a, an, and, as, at, but, by, for, in, nor, of, on, or, the, to
% and up, which are usually not capitalized unless they are the first or
% last word of the caption. Table text will default to \footnotesize as
% the IEEE normally uses this smaller font for tables.
% The \label must come after \caption as always.
%
%\begin{table}[!t]
%% increase table row spacing, adjust to taste
%\renewcommand{\arraystretch}{1.3}
% if using array.sty, it might be a good idea to tweak the value of
% \extrarowheight as needed to properly center the text within the cells
%\caption{An Example of a Table}
%\label{table_example}
%\centering
%% Some packages, such as MDW tools, offer better commands for making tables
%% than the plain LaTeX2e tabular which is used here.
%\begin{tabular}{|c||c|}
%\hline
%One & Two\\
%\hline
%Three & Four\\
%\hline
%\end{tabular}
%\end{table}


% Note that the IEEE does not put floats in the very first column
% - or typically anywhere on the first page for that matter. Also,
% in-text middle ("here") positioning is typically not used, but it
% is allowed and encouraged for Computer Society conferences (but
% not Computer Society journals). Most IEEE journals/conferences use
% top floats exclusively. 
% Note that, LaTeX2e, unlike IEEE journals/conferences, places
% footnotes above bottom floats. This can be corrected via the
% \fnbelowfloat command of the stfloats package.




\section{Conclusion}
The conclusion goes here.




% conference papers do not normally have an appendix


% use section* for acknowledgment
\section*{Acknowledgment}


The authors\cite{nastikya} would like to thank...
\renewcommand{\bibfont}{\normalfont\small}
\printbibliography




% that's all folks
\end{document}


