\section{Introduction}
Text classification is the task of assigning a text document into a set of predefined classes in an intelligent manner. Because of the rapid growth of online information, text classification has become more challenging and more important as well. Digitization has changed
the way we process and analyze information. There is an exponential increase in online availability of information. From web pages to emails, science journals, ebooks, learning content, news and social media are all full of textual data. Text classification performs an essential role in various applications that deals with organizing, classifying, searching and concisely representing a significant amount of information. Detecting suspicious text is typically a text classification problem where we have to classify a text as suspicious and not suspicious. Suspicious text detection is a kind of system where suspicious texts are identified by the keywords used in the text body. As most of our communications are text based, if we able to predict either a text is suspicious or not suspicious it will be very helpful for our law enforcement agencies to find the perpetrator and stop terrorist event. As far we know, no system is developed for detecting suspicious Bangla text. It is very important for the safety of Bangladeshi People to develop a system which can detect suspicious communications in Bangla. This motivated us to work in this area.

In this work, our original purpose is to develop a framework for detecting suspicious Bangla texts using supervised machine learning techniques. In this paper, we have applied Naïve Bayes Classifier\cite{yoo2015classification}, Support Vector Machine\cite{wei2012text, villmann2017can}, Logistic regression\cite{sharma2015active}, K-nearest neighbor algorithms (KNN)\cite{harisinghaney2014text}, Decision Trees\cite{chavan2014survey} to detect suspicious Bangla text and also find out the accuracy of this algorithms.